%!TEX root = Style.tex
%\pagestyle{plain}
\chapter*{Обзор существующих инструментов}
\setcounter{chapter}{1}
\addcontentsline{toc}{chapter}{Обзор существующих инструментов}%

\section{Izitru}

\textbf{URL: } \url{http://www.izitru.com/}

\textbf{Авторы:}

Kevin Connor, a former vice president of product management for Photoshop,

and Dr.Hany Farid, a leading image forensics expert.

\textbf{Что это:} Сайт

\textbf{Стоимость:} Бесплатное

\textbf{Поддерживаемый формат файлов:} JPEG

\textbf{Используемые технологии и алгоритмы:}

\begin{enumerate}
  \item \href{http://fotoforensics.com/tutorial-meta.php}{Metadata analysis}
  \item Device signature analysis
  
  \label{itm:DCAA}
  \item \hypertarget{itm:DCAB}{\hyperlink{itm:DCAA}{Double JPEG compression detection analysis}}
  
  \item JPEG DCT coefficiets analysis (or Double Quantization Effect analysis)
  
  \item Sensor pattern analysis
  
  \item JPEG ghosts analysis
\end{enumerate} 

\textbf{Удобство интерфейса:}
Ясно как запустить анализ изображения, ясен и результат анализа в целом (программа выводит “диагноз” изображения). Результаты же каждого метода анализа данного инструмента не выводятся.


\textbf{Минусы:}
\begin{enumerate}
  \item	Не все метаданные анализирует.
\end{enumerate}
  
\textbf{Дополнительные характеристики:}
\begin{enumerate}
  \item	Есть IPhone app.
\end{enumerate}

\textbf{Тестирование на изображениях с модификацией:}
\begin{enumerate}
\item \textbf{Оригинальный JPEG с камеры:}
Подтверждает, что это немодифицированное изображение, полученное с камеры.
\url{https://www.izitru.com/39jlW}
\item \textbf{Оригинальный снимок Raw с камеры, пересохраненный в JPEG компьютерной программой:}
Выводит подозрение,что файл мог быть модифицированным.
\url{https://www.izitru.com/iymM2}
\item \textbf{Размытие, растяжение, изменение контраста, клонирование объекта, вставка другого изображения, применение фильтра “аппликация”, добавление блика:}
Реагирует также, как и на просто пересохранение файла в JPEG в фоторедакторе.
\end{enumerate}


\textbf{Комментарий:}
Вообще, этот сайт позиционирует себя, как инструмент для доказательства, что фотография не была модифицирована, а не как инструмент для определения модификаций на фотографиях. Он включен в этот список, т.к. многие статьи описывают его, как инструмент для определения модифицировано ли изображение.


\newpage

\section{Image edited?}

\textbf{URL: } \url{http://imageedited.com/}

\textbf{Авторы:}

 Неизвестны

\textbf{Что это:} Сайт

\textbf{Стоимость:} Бесплатное, но есть и платная версия

\textbf{Поддерживаемый формат файлов:} JPEG

\textbf{Используемые технологии и алгоритмы:}

\begin{enumerate}
  \item \href{http://fotoforensics.com/tutorial-meta.php}{Metadata analysis}
  \item Error Level Analysis (ELA)
  
  \item Color Distribution Discontinuities Analysis
  
  \item Color Oversaturation Analysis
  
  \item Sharpness Grain Analysis
  
  \item Light Direction Mismatch analysis
\end{enumerate} 

\textbf{Удобство интерфейса:}
Ясно как запустить анализ изображения, ясен и результат анализа в целом (программа выводит “диагноз” изображения). 

\textbf{Минусы:}
\begin{enumerate}
  \item	В бесплатной версии отображается только информация о метаданных.
\end{enumerate}

\textbf{Тестирование на изображениях с модификацией:}
\begin{enumerate}
\item \textbf{Оригинальный JPEG с камеры:}
Подтверждает, что это возможно снимок, полученный с камеры.
\item \textbf{Оригинальный снимок Raw с камеры, пересохраненный в JPEG компьютерной программой:}
Если компьютерная программа оставила свои тэги в мета данных, то сайт утверждает, что изображение было модифицировано. Если же не оставила, то выводит сообщение, что пиксели изображения соответствуют фоторедактору.

\item \textbf{Размытие, растяжение, изменение контраста, клонирование объекта, вставка другого изображения, применение фильтра “аппликация”, добавление блика:}
Реагирует также, как и на просто пересохранение файла в JPEG в фоторедакторе.
\end{enumerate}

\newpage

\section{Picture manipulation inspector}

\textbf{URL: } \url{http://www.smtdp.com/ru/products/}

\textbf{Авторы:}

Компания «SMTDP»

\textbf{Что это:} Сайт

\textbf{Стоимость:} Бесплатно только 3 фотографии в день.

\textbf{Поддерживаемый формат файлов:} JPEG

\textbf{Используемые технологии и алгоритмы:}

\begin{enumerate}
  \item \href{http://fotoforensics.com/tutorial-meta.php}{Metadata analysis}
  \item Quantization Table Analysis
  
  \item Double Quantization Effect analysis
  
  \item Analysis of response function of the camera 
  
  \item Clone Detection Analysis
  
  \item Double Compression Artifacts Analysis
  
  \item Error Level Analysis (ELA)
\end{enumerate} 

\textbf{Удобство интерфейса:}
Ясно как запустить анализ изображения, ясен и результат анализа в целом (программа выводит уровень доверия к изображению).

Инструмент также выводит некоторые результаты анализа в виде картинок без пояснений, которые для людей, не разбирающихся в методах анализа, не понятны.
  
\textbf{Дополнительные характеристики:}
\begin{enumerate}
  \item	Производительность: 100 изображений в минуту.
  \item В базе данных есть информация о 3000 моделей фотоаппаратов и мобильных устройств с камерами (Формат мета данных, стандартная таблица квантования для jpg, параметры качества сжатия jpg).
\end{enumerate}

\textbf{Тестирование на изображениях с модификацией:}
\begin{enumerate}
\item \textbf{Оригинальный JPEG с камеры:}
Выводит ложную информацию о том, что этот снимок открывался программой Photoshop и отображает очень низкий уровень доверия к этой фотографии.
\url{http://www.smtdp.com/ru/free-demo/?insp=5802ef3a975ab229b34d06cb}

Такой же результат был и с другими оригинальными изображениями с моей камеры.
\item \textbf{Размытие:}
Инструмент не выводит сообщений о модификации “размытие” и её области применения. Визуально определить эту модификацию по картинкам, приведенным инструментом, по-моему мнению, возможно.
\url{http://www.smtdp.com/ru/free-demo/?insp=5803377c975ab229b34d06cd}

\item \textbf{Растяжение:}
Инструмент не выводит сообщений о модификации “растяжение” и её области применения. Визуально определить эту модификацию по картинкам, приведенным инструментом, по-моему мнению, сложновато.
\url{http://www.smtdp.com/ru/free-demo/?insp=5803381d975ab229b34d06ce}

\item \textbf{Изменение контраста:}
Инструмент не выводит сообщений о модификации “изменение контраста” и её области применения. Визуально определить эту модификацию по картинкам, приведенным инструментом, по-моему мнению, сложновато.
\url{http://www.smtdp.com/ru/free-demo/?insp=5803388a975ab229b34d06cf}

\item \textbf{Клонирование объекта:}
Инструмент не выводит сообщений о модификации “клонирование объекта” и её области применения. Визуально определить эту модификацию по картинкам, приведенным инструментом, по-моему мнению, сложновато.
\url{http://www.smtdp.com/free-demo/?insp=580339fb975ab229b34d06d0}

\item \textbf{Вставка другого изображение:}
Инструмент не выводит сообщений о модификации “вставка другого изображения” и её области применения. Визуально определить эту модификацию по картинкам, приведенным инструментом, по-моему можно.
\url{http://www.smtdp.com/free-demo/?insp=58033a68975ab229b34d06d1} 

\item \textbf{Применения фильтра "aппликация":}
Инструмент не выводит сообщений о применении фильтра “аппликация” и его области применения. Визуально определить эту модификацию по картинкам, приведенным инструментом, по-моему нельзя, можно только понять, что все изображение подверглось какой-то модифакации. 
\url{http://www.smtdp.com/free-demo/?insp=58033acb975ab229b34d06d2}
\end{enumerate}

\newpage

\section{Image forensic (Ghiro)}

\textbf{URL: } \url{http://www.imageforensic.org/}

\textbf{Авторы:}

Alessandro Tanasi, Marco Buoncristiano

\textbf{Что это:} Сайт

\textbf{Стоимость:} Бесплатное

\textbf{Поддерживаемый формат файлов:}
\begin{itemize}
\item Windows bitmap .bmp
\item Raw Canon .cr2
\item Raw Canon .crw
\item Encapsulated PostScript .eps
\item Graphics Interchange Format .gif
\item JPEG File Interchange Format .jpg or .jpeg
\item Raw Minolta .mrw
\item Raw Olympus .orf
\item Portable Network Graphics .png
\item Raw Photoshop .psd
\item Raw Fujifilm .raf
\item Raw Panasonic .rw2
\item Raw TARGA .tga
\item Tagged Image File Format .tiff
\end{itemize} 

\textbf{Используемые технологии и алгоритмы:}

\begin{enumerate}
  \item \href{http://fotoforensics.com/tutorial-meta.php}{Metadata analysis}
  \item Thumbnail analysis
  
  \item Device signature analysis
  
  \item Error Level Analysis (ELA)
\end{enumerate} 

\textbf{Удобство интерфейса:}
Ясно как запустить анализ изображения, но не ясен результат анализа в целом (программа не выводит никакого конечного “диагноза” изображения).
Инструмент также выводит некоторые результаты анализа в виде картинок без пояснений, которые для людей, не разбирающихся в методах анализа, не понятны.
Также неочевидно, что если кликнуть по некоторым полям, то они отобразят дополнительную информацию.
Инструмент также выводит некоторые результаты анализа в виде картинок без пояснений, которые для людей, не разбирающихся в методах анализа, не понятны.
  
\textbf{Дополнительные характеристики:}
\begin{enumerate}
  \item Open source.
  \item Обеспечивает приватность.
  \item Создатели просят не нагружать сервер большим количеством запросов.
  \item Подсчитывает разные хэш-суммы для идентификации изображения (CRC32, MD5, SHA1, SHA224, SHA256, SHA384, SHA512)
\end{enumerate}

\textbf{Тестирование на изображениях с модификацией:}
\begin{enumerate}
\item \textbf{Оригинальный JPEG с камеры:}

\url{http://www.imageforensic.org/show/b7e3d482bae3f153335ccfc199167b1a/64bfb135-df57-48c4-90d0-63fa70e0cc0c}

\item \textbf{Оригинальный снимок Raw с камеры, пересохраненный в JPEG компьютерной программой:}
Если изображение было сохранено в Adobe Photoshop, то можно будет найти мета данные от Adobe Photoshop.
\url{http://www.imageforensic.org/show/87ee14df942b2ed6565b7a630e56440d/8084079f-c950-40e3-a14d-4dfce6718646}
Если изображение было сохранено какой-то другой программой, то можно и не найти никаких мета данных в принципе.
\url{http://www.imageforensic.org/show/5ed0b0e529b98560ecb8419a8f31e296/c5184155-3527-4726-be50-93479c6df7c1}

\item \textbf{Размытие:}
Инструмент не выводит сообщений о модификации “размытие” и её области применения. Визуально определить эту модификацию по картинкам, приведенным инструментом, по-моему мнению, возможно.
\url{http://www.imageforensic.org/show/3f24a2bcc6c4ec1e1562c1719743256e/0811a397-1142-4883-b72a-0c8e3428f400}

\item \textbf{Растяжение:}
Инструмент не выводит сообщений о модификации “растяжение” и её области применения. Визуально определить эту модификацию по картинкам, приведенным инструментом, по-моему мнению, сложновато.
\url{http://www.imageforensic.org/show/9fd6ef2f0f0a0600d37e6b78e4d1673d/c2c088b6-08a0-4cf9-aeab-23550fa5430d}

\item \textbf{Изменение контраста:}
Инструмент не выводит сообщений о модификации “изменение контраста” и её области применения. Визуально определить эту модификацию по картинкам, приведенным инструментом, по-моему мнению, сложновато.
\url{http://www.imageforensic.org/show/90821eee67a902991678f5bf46934a78/c07e563a-522b-45f2-8726-4e389679737e}

\item \textbf{Клонирование объекта:}
Инструмент не выводит сообщений о модификации “клонирование объекта” и её области применения. Визуально определить эту модификацию по картинкам, приведенным инструментом, по-моему мнению, возможно.
\url{http://www.imageforensic.org/show/462713f78881339c99c6f8971a1b9125/a85ace1a-334c-429c-a85a-688b3d96daa2}

\item \textbf{Вставка другого изображение:}
Инструмент не выводит сообщений о модификации “вставка другого изображения” и её области применения. Визуально определить эту модификацию по картинкам, приведенным инструментом, по-моему мнению, возможно.
\url{http://www.imageforensic.org/show/a704bc9a5abe1adeda4cbba08f4d37d3/7fa91699-758b-4d3c-a99b-ce89ca3c0253}

\item \textbf{Применения фильтра "aппликация":}
Визуально, по-моему мнению, можно только понять, что все изображение подверглось какой-то модифакации. 
\url{http://www.imageforensic.org/show/8e4ae85b24b875e9fe5441d80821e8c7/a38d477d-ab00-43d5-be08-b52f28a29e95}

\item \textbf{Добавление блика:}
Инструмент не выводит сообщений о модификации “Добавление блика ” и её области применения. Визуально определить эту модификацию по картинкам, приведенным инструментом, по-моему сложновато.
\url{http://www.imageforensic.org/show/851da3b7de64f79fbfbe4b148adf85e7/51a0fbde-9deb-4e79-b4e0-f9ab66b12b65}

\end{enumerate}

\newpage

\section{Image forgery detector}

\textbf{URL: } \url{http://ifdetector.com/}

\textbf{Авторы:}

Scorto Corporation, PhDs in Mathematics, Statistics and Artificial Intelligence

\textbf{Что это:} Сайт

\textbf{Стоимость:} Платное, но есть бесплатная демо-версия, которую по каким-то причинам мне не выслали.

\textbf{Поддерживаемый формат файлов:} Неизвестно

\textbf{Используемые технологии и алгоритмы:}

\begin{enumerate}
  \item Современные алгоритмы анализа изображения (методы не описаны)
  \item Алгоритмы машинного обучения (методы не описаны)
\end{enumerate} 

\textbf{Удобство интерфейса:}
Судя по скринам, ясно как запустить анализ изображения. Понятен и в целом результат, но как получился такой результат непонятно.
  
\newpage

\section{Fotoforensics}

\textbf{URL: } \url{http://fotoforensics.com/}

\textbf{Авторы:}

HackerFactor

\textbf{Что это:} Сайт

\textbf{Стоимость:} Есть платные и бесплатные версии

\textbf{Поддерживаемый формат файлов:} JPEG, PNG

\textbf{Используемые технологии и алгоритмы:}

\begin{enumerate}
  \item \href{http://fotoforensics.com/tutorial-meta.php}{Metadata analysis}
  \item Error Level Analysis (ELA)
    
  \item Similar Picture Search
  
  \item Color Adjustment Analysis
\end{enumerate} 

\textbf{Дополнительные технологии и алгоритмы в платной версии:}
\begin{enumerate}
  \item Thumbnail analysis
\end{enumerate} 

\textbf{Удобство интерфейса:}
Ясно как запустить анализ изображения, но не ясен результат анализа в целом (программа не выводит никакого конечного “диагноза” изображения).
Инструмент также выводит некоторые результаты анализа в виде картинок без пояснений, которые для людей, не разбирающихся в методах анализа, не понятны
  
\textbf{Дополнительные характеристики:}
\begin{enumerate}
  \item Подсчитывает хэш-суммы для идентификации изображения.
  \item Estimate JPEG Quality.
  \item Можно проводить некоторые трансформации изображения.
\end{enumerate}

\textbf{Тестирование на изображениях с модификацией:}
\begin{enumerate}
\item \textbf{Оригинальный JPEG с камеры:}
\url{http://fotoforensics.com/analysis.php?id=4c2765bd8b345ef980e3684cc2576a720eabd6c3.5785321}

\item \textbf{Оригинальный снимок Raw с камеры, пересохраненный в JPEG компьютерной программой:}
Программа вывела сообщение о том, что файл слишком большого размера, чтобы анализировать (размер файла 9,34 МБ).

\item \textbf{Размытие:}
Инструмент не выводит сообщений о модификации “размытие” и её области применения. Визуально определить эту модификацию по картинкам, приведенными инструментом, по-моему мнению, возможно.
\url{http://fotoforensics.com/analysis.php?id=cdb64098f351cce5a8bce7d5e4d9c3ab1517681a.6843023}

\item \textbf{Растяжение:}
Инструмент не выводит сообщений о модификации “растяжение” и её области применения. Визуально определить эту модификацию по картинкам, приведенными инструментом, по-моему мнению, сложновато.
\url{http://fotoforensics.com/analysis.php?id=df3b11995853ec206a714c2edb3660243e53c6be.7125650}

\item \textbf{Изменение контраста:}
Инструмент не выводит сообщений о модификации “изменение контраста” и её области применения. Визуально определить эту модификацию по картинкам, приведенными инструментом, по-моему мнению, сложновато.
\url{http://fotoforensics.com/analysis.php?id=035bb71b508332395f9e3ef10f3a214110bd642a.8126211}

\item \textbf{Клонирование объекта:}
Инструмент не выводит сообщений о модификации “клонирование объекта” и её области применения. Визуально определить эту модификацию по картинкам, приведенными инструментом, по-моему мнению, возможно.
\url{http://fotoforensics.com/analysis.php?id=888b530e129013fdaf0d248e72a8a85a4016c4d9.7219430}

\item \textbf{Вставка другого изображение:}
Инструмент не выводит сообщений о модификации “вставка другого изображения” и её области применения. Визуально определить эту модификацию по картинкам, приведенными инструментом, по-моему мнению, можно.
\url{http://fotoforensics.com/analysis.php?id=93f87eb0ed63027e9f476621eb1b5507ca1d5b4a.7125426}

\item \textbf{Применения фильтра "aппликация":}
Визуально, по-моему мнению, можно только понять, что все изображение подверглось какой-то модифакации. 
\url{http://fotoforensics.com/analysis.php?id=8456b554c0e16c56341960b83747e6a2525d4210.1408295}

\item \textbf{Добавление блика:}
Инструмент не выводит сообщений о модификации “добавление блика” и её области применения. Визуально определить эту модификацию по картинкам, приведенными инструментом, по-моему мнению, возможно.
\url{http://fotoforensics.com/analysis.php?id=48a46a1ef58017565ed75fb58ec3c6249de4cf56.7092325}

\end{enumerate}

\newpage

\section{Photo detective}

\textbf{URL: } \url{http://metainventions.com/photodetective.html}

\textbf{Авторы:}

MetaInventions is a Chicago-based research and software development lab.

\textbf{Что это:} Настольная программа

\textbf{Поддерживаемые платформы:} Windows XP, Vista, 7, 8, 10 (32-bit and 64-bit)

\textbf{Стоимость:} Платное

\textbf{Поддерживаемый формат файлов:} JPEG, PNG

\textbf{Используемые технологии и алгоритмы:}

\begin{enumerate}
  \item Principal Component Analysis to detect consistency among artifacts in the image
  \item Wavelet decomposition analysis (focal length analysis)
    
  \item Edge detection routines to identify naturally blurred vs. artificially blurred areas (used to covered up tampering)
  
  \item Error Level Analysis (ELA)
  
  \item Local Min/Max pixel highlighting to determine if the pixels errors display a normal variation
  \item Lighting gradient coloration analysis
  \item Temperature Pseudocolor based on pixel intensity to identify odd or inconsistent coloration
  \item Pixel color randomization to show pixels that have the same RGB value more easily.
  \item Extreme color highlighting to show where the pixels are pure white and black and if the colors are being clipped by the camera
    \item Thumbnail analysis
    \item Metadata analysis
	\item JPEG quantization tables analysis
    \item Minimum-Medium-Maximum RGB value for each pixel to identify coloration inconsistencies 
    \item Histogram of colors analysis
	\item Color Filter Array Estimation analysis
\end{enumerate} 

\textbf{Дополнительные технологии и алгоритмы в платной версии:}
\begin{enumerate}
  \item Thumbnail analysis
\end{enumerate} 

\textbf{Удобство интерфейса:}
Не имела возможности оценить. 

\newpage

\section{Forensically}

\textbf{URL: } \url{ https://29a.ch/photo-forensics/#forensic-magnifier}

\textbf{Авторы:}

Jonas Wagner is a software engineer based in Zurich, Switzerland

\textbf{Что это:} Сайт

\textbf{Стоимость:} Бесплатное

\textbf{Поддерживаемый формат файлов:} JPEG, PNG и мб какие-то еще

\textbf{Используемые технологии и алгоритмы:}

\begin{enumerate}
  \item \href{http://fotoforensics.com/tutorial-meta.php}{Metadata analysis}
  \item Principal component analysis (PCA)
    
  \item Error Level Analysis (ELA)
  
  \item Luminance gradient analysis
  
  \item Level sweep
  
  \item Clone detection analysis
  
  \item Thumbnail analysis
\end{enumerate} 

\textbf{Дополнительные технологии и алгоритмы в платной версии:}
\begin{enumerate}
  \item Thumbnail analysis
\end{enumerate} 

\textbf{Удобство интерфейса:}
Ясно как запустить анализ изображения, но не ясен результат анализа в целом (программа не выводит никакого конечного “диагноза” изображения).
Результат применения каждого инструмента анализа программы приходится анализировать визуально.

\textbf{Тестирование на изображениях с модификацией:}
\begin{enumerate}

\item \textbf{Размытие:}
Инструмент не выводит сообщений о модификации “размытие” и её области применения. Визуально определить эту модификацию по картинкам, приведенными инструментом, по-моему мнению, можно.

\item \textbf{Растяжение:}
Инструмент не выводит сообщений о модификации “растяжение” и её области применения. Визуально определить эту модификацию по картинкам, приведенными инструментом, по-моему мнению, возможно.

\item \textbf{Изменение контраста:}
Инструмент не выводит сообщений о модификации “изменение контраста” и её области применения. Визуально определить эту модификацию по картинкам, приведенными инструментом, по-моему мнению, сложновато.

\item \textbf{Клонирование объекта:}
Инструмент не выводит сообщений о модификации “клонирование объекта” и её области применения. Визуально определить эту модификацию по картинкам, приведенными инструментом, по-моему мнению, можно.

\item \textbf{Вставка другого изображение:}
Инструмент не выводит сообщений о модификации “вставка другого изображения” и её области применения. Визуально определить эту модификацию по картинкам, приведенными инструментом, по-моему мнению, можно.

\item \textbf{Применения фильтра "aппликация":}
Визуально, по-моему мнению, можно только понять, что все изображение подверглось какой-то модифакации. 

\item \textbf{Добавление блика:}
Инструмент не выводит сообщений о модификации “вставка другого изображения” и её области применения. Визуально определить эту модификацию по картинкам, приведенными инструментом, по-моему мнению, возможно.

\end{enumerate}

\newpage

\section{Authenticate}

\textbf{URL: } \url{https://ampedsoftware.com/authenticate}

\textbf{Авторы:}

Amped is a company specialized in developing software solutions for image and video processing for forensic and investigative applications.

\textbf{Что это:} Настольная программа

\textbf{Поддерживаемые платформы:} Windows XP, Vista, 7, 8 (32-bit and 64-bit) 

\textbf{Стоимость:} Платное

\textbf{Поддерживаемый формат файлов:} JPEG, TIFF, BMP, PNG and raw format from digital cameras.
\textbf{Используемые технологии и алгоритмы:}

\begin{enumerate}
  \item \href{http://fotoforensics.com/tutorial-meta.php}{Metadata analysis}
  \item JPEG quantization  tables analysis
    
  \item Thumbnail analysis
  
  \item Correlation Plot
  
  \item JPEG Ghosts
  
  \item Histogram of the image analysis
  
  \item Analysis of the color space usage of the image in the HSV and Lab coordinates that can help spot excessive color adjustment
  
  \item PRNU Identification
  
  \item Analysis of single image channels in different color spaces (RGB, YCbCr, YUV, HSV, HLS, XYZ, Lab, Luv, CMYK)
  
  \item Error level analysis (ELA)
    
  \item Display of the image DCT values that can help to spot tampered uniform areas of the image
  
  \item Identification of discontinuities in the correlation between pixels of the image
  
  \item Noise level analysis
  
  \item Luminance gradient analysis

  \item Clone detection
  
  \item Similar Picture Search
  
\end{enumerate} 

\textbf{Удобство интерфейса:}
Судя по скринам, ясно как запустить анализ изображения. Насколько понятен результат анализа оценить не могу.
  
\textbf{Дополнительные характеристики:}
\begin{enumerate}
  \item Изъятие изображения из файлов формата PDF, PPT, DOC 
\end{enumerate}

\newpage
\section{Описание некоторых методов анализа изображений}
\begin{enumerate}
  \item JPEG Quantization Tables Analysis:
Цифровые фотоаппараты, мобильные девайсы и программы для редактирования
изображений используют различные таблицы квантования при сохранении изображений в
формат JPEG. Таблицы квантования могут быть извлечены и проанализированы. Если
таблицы отличаются от тех, которые использует модель фотоаппарата, указанного в EXIF
данных изображения, значит с изображением проводили какие-либо манипуляции.
  \label{itm:DCAA}
  \item \hypertarget{itm:DCAA}{\hyperlink{itm:DCAB}{Double Compression Artifacts Analysis:}} 
При повторном сохранении файла формата JPEG происходит повторное сжатие, в
результате этого коэффициент сжатия увеличивается. При большом коэффициенте
сжатия появляются артефакты сжатия JPEG, которые представляют собой
прямоугольные поля одного цвета, которые могут принимать довольно крупный размер в
одноцветных областях изображения. Данный анализ исследует изображение на
артефакты сжатия JPEG.
\end{enumerate}