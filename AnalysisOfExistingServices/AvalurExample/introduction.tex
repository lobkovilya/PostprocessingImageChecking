%!TEX root = diplom.tex
%\pagestyle{plain}
\chapter*{Обзор существующих инструментов}
\setcounter{chapter}{1}
\addcontentsline{toc}{chapter}{Обзор существующих инструментов}%

\section{Izitru}

\textbf{URL: } \url{http://www.izitru.com/}

\textbf{Авторы:}

Kevin Connor, a former vice president of product management for Photoshop, and Dr.

Hany Farid, a leading image forensics expert.

\textbf{Что это:} Сайт

\textbf{Стоимость:} Бесплатное

\textbf{Поддерживаемый формат файлов:} JPEG

\textbf{Используемые технологии и алгоритмы:}
\begin{enumerate}
  \item \href{http://fotoforensics.com/tutorial-meta.php}{Metadata analysis}
  \item Device signature analysis
  \item \emph{Можно так \hyperlink{itm:DCAA}{Double JPEG compression detection analysis} или так \autoref{itm:DCAA} сослаться. Ну обратные ссылки тоже, но только 1 к 1 функция перехода на нужное место в файле.}
\end{enumerate}

\newpage
\section{Описание некоторых методов анализа изображений}
\begin{enumerate}
  \item JPEG Quantization Tables Analysis:
Цифровые фотоаппараты, мобильные девайсы и программы для редактирования
изображений используют различные таблицы квантования при сохранении изображений в
формат JPEG. Таблицы квантования могут быть извлечены и проанализированы. Если
таблицы отличаются от тех, которые использует модель фотоаппарата, указанного в EXIF
данных изображения, значит с изображением проводили какие-либо манипуляции.
  \label{itm:DCAA}
  \item \hypertarget{itm:DCAA}{Double Compression Artifacts Analysis:} 
При повторном сохранении файла формата JPEG происходит повторное сжатие, в
результате этого коэффициент сжатия увеличивается. При большом коэффициенте
сжатия появляются артефакты сжатия JPEG, которые представляют собой
прямоугольные поля одного цвета, которые могут принимать довольно крупный размер в
одноцветных областях изображения. Данный анализ исследует изображение на
артефакты сжатия JPEG.
\end{enumerate}
